% Options for packages loaded elsewhere
\PassOptionsToPackage{unicode}{hyperref}
\PassOptionsToPackage{hyphens}{url}
%
\documentclass[
  12pt,
]{article}
\usepackage{amsmath,amssymb}
\usepackage{lmodern}
\usepackage{ifxetex,ifluatex}
\ifnum 0\ifxetex 1\fi\ifluatex 1\fi=0 % if pdftex
  \usepackage[T1]{fontenc}
  \usepackage[utf8]{inputenc}
  \usepackage{textcomp} % provide euro and other symbols
\else % if luatex or xetex
  \usepackage{unicode-math}
  \defaultfontfeatures{Scale=MatchLowercase}
  \defaultfontfeatures[\rmfamily]{Ligatures=TeX,Scale=1}
\fi
% Use upquote if available, for straight quotes in verbatim environments
\IfFileExists{upquote.sty}{\usepackage{upquote}}{}
\IfFileExists{microtype.sty}{% use microtype if available
  \usepackage[]{microtype}
  \UseMicrotypeSet[protrusion]{basicmath} % disable protrusion for tt fonts
}{}
\makeatletter
\@ifundefined{KOMAClassName}{% if non-KOMA class
  \IfFileExists{parskip.sty}{%
    \usepackage{parskip}
  }{% else
    \setlength{\parindent}{0pt}
    \setlength{\parskip}{6pt plus 2pt minus 1pt}}
}{% if KOMA class
  \KOMAoptions{parskip=half}}
\makeatother
\usepackage{xcolor}
\IfFileExists{xurl.sty}{\usepackage{xurl}}{} % add URL line breaks if available
\IfFileExists{bookmark.sty}{\usepackage{bookmark}}{\usepackage{hyperref}}
\hypersetup{
  pdftitle={There Votes the Neighborhood: Gentrification, Displacement, and Political Participation},
  pdfauthor={Kevin Morris},
  hidelinks,
  pdfcreator={LaTeX via pandoc}}
\urlstyle{same} % disable monospaced font for URLs
\usepackage[margin=1in]{geometry}
\usepackage{longtable,booktabs,array}
\usepackage{calc} % for calculating minipage widths
% Correct order of tables after \paragraph or \subparagraph
\usepackage{etoolbox}
\makeatletter
\patchcmd\longtable{\par}{\if@noskipsec\mbox{}\fi\par}{}{}
\makeatother
% Allow footnotes in longtable head/foot
\IfFileExists{footnotehyper.sty}{\usepackage{footnotehyper}}{\usepackage{footnote}}
\makesavenoteenv{longtable}
\usepackage{graphicx}
\makeatletter
\def\maxwidth{\ifdim\Gin@nat@width>\linewidth\linewidth\else\Gin@nat@width\fi}
\def\maxheight{\ifdim\Gin@nat@height>\textheight\textheight\else\Gin@nat@height\fi}
\makeatother
% Scale images if necessary, so that they will not overflow the page
% margins by default, and it is still possible to overwrite the defaults
% using explicit options in \includegraphics[width, height, ...]{}
\setkeys{Gin}{width=\maxwidth,height=\maxheight,keepaspectratio}
% Set default figure placement to htbp
\makeatletter
\def\fps@figure{htbp}
\makeatother
\setlength{\emergencystretch}{3em} % prevent overfull lines
\providecommand{\tightlist}{%
  \setlength{\itemsep}{0pt}\setlength{\parskip}{0pt}}
\setcounter{secnumdepth}{5}
\usepackage{rotating}
\usepackage{setspace}
\newcommand{\beginsupplement}{\setcounter{table}{0}  \renewcommand{\thetable}{A\arabic{table}} \setcounter{figure}{0} \renewcommand{\thefigure}{A\arabic{figure}}}
\usepackage{lineno}
\usepackage{booktabs}
\usepackage{longtable}
\usepackage{array}
\usepackage{multirow}
\usepackage{wrapfig}
\usepackage{float}
\usepackage{colortbl}
\usepackage{pdflscape}
\usepackage{tabu}
\usepackage{threeparttable}
\usepackage{threeparttablex}
\usepackage[normalem]{ulem}
\usepackage{makecell}
\usepackage{xcolor}
\ifluatex
  \usepackage{selnolig}  % disable illegal ligatures
\fi
\newlength{\cslhangindent}
\setlength{\cslhangindent}{1.5em}
\newlength{\csllabelwidth}
\setlength{\csllabelwidth}{3em}
\newenvironment{CSLReferences}[2] % #1 hanging-ident, #2 entry spacing
 {% don't indent paragraphs
  \setlength{\parindent}{0pt}
  % turn on hanging indent if param 1 is 1
  \ifodd #1 \everypar{\setlength{\hangindent}{\cslhangindent}}\ignorespaces\fi
  % set entry spacing
  \ifnum #2 > 0
  \setlength{\parskip}{#2\baselineskip}
  \fi
 }%
 {}
\usepackage{calc}
\newcommand{\CSLBlock}[1]{#1\hfill\break}
\newcommand{\CSLLeftMargin}[1]{\parbox[t]{\csllabelwidth}{#1}}
\newcommand{\CSLRightInline}[1]{\parbox[t]{\linewidth - \csllabelwidth}{#1}\break}
\newcommand{\CSLIndent}[1]{\hspace{\cslhangindent}#1}

\title{There Votes the Neighborhood: Gentrification, Displacement, and Political Participation}
\author{Kevin Morris\footnote{PhD Student, CUNY Graduate Center, Department of Sociology (\href{mailto:kmorris@gradcenter.cuny.edu}{\nolinkurl{kmorris@gradcenter.cuny.edu}})}}
\date{September 14, 2021}

\begin{document}
\maketitle
\begin{abstract}
TKTKTK
\end{abstract}

\pagenumbering{gobble}
\pagebreak

\pagenumbering{arabic}
\doublespacing

\hypertarget{introduction}{%
\section*{Introduction}\label{introduction}}
\addcontentsline{toc}{section}{Introduction}

\hypertarget{gentrification-as-a-state-practice}{%
\section*{Gentrification as a State Practice}\label{gentrification-as-a-state-practice}}
\addcontentsline{toc}{section}{Gentrification as a State Practice}

The latter half of the 20th century for the American metropolis was characterized by urban flight, disinvestment, and hyper-segregation. In the aftermath of the 1965 civil rights movement, many white Americans fled urban areas, both leading and following jobs out of these areas (\protect\hyperlink{ref-Jackson1985}{Jackson 1985}; \protect\hyperlink{ref-Sugrue1998}{Sugrue 1998}). While middle class white Americans were able to decamp for suburban areas, even relatively wealthy Black Americans were forced to remain in decaying urban neighborhoods thanks to racial steering from the real estate industry and racist lending practices by financial institutions (\protect\hyperlink{ref-Rothstein2017}{Rothstein 2017}). \protect\hyperlink{ref-Massey2003}{Massey and Denton} (\protect\hyperlink{ref-Massey2003}{2003}) documents many of the implications of this \emph{de facto} segregation, demonstrating the vulnerability of Black Americans---regardless of class---to economic adversity. As \protect\hyperlink{ref-Taylor2019}{Taylor} (\protect\hyperlink{ref-Taylor2019}{2019}) shows, the consequences of these restrictive lending practices did more than simply keep Black Americans out of the suburbs: they also inhibited Black residents from buying property in their neighborhoods or investing in property that they did own. Declining tax bases and waning political capital meant that public services in urban areas deteriorated quickly, hastening the out migration of those who could do so (see, for instance, \protect\hyperlink{ref-Phillips-Fein2017}{Phillips-Fein 2017}).

In the late 20th and early 21st centuries, these patterns reversed as Americans began moving back to some urban areas. Increased capital investments (\protect\hyperlink{ref-Birch2009}{Birch 2009}) formed an important part of this back-to-the-city movement (\protect\hyperlink{ref-Hyra2015}{Hyra 2015}). As increasing numbers of highly-educated and highly-compensated Americans sought to live in cities such as San Francisco and New York, communities that had been marginalized (both economically and physically) for a generation or more found themselves residing in areas with increasing potential value. As Samuel Stein argues in \emph{Capital City} (\protect\hyperlink{ref-Stein2019}{2019}), financial capitalists recognized the mismatch between the current and potential income being produced in urban neighborhoods. He explains: ``Real estate speculators choose to invest in a particular location because they identify a gap between the rents that land currently offers and the potential future rents it might command if some action were taken'' (\protect\hyperlink{ref-Stein2019}{Stein 2019, 49}). Stein argues, however, that gentrification is at heart a political issue; it cannot transpire without the state clearing the way for investors to exploit that value gap. Smith (\protect\hyperlink{ref-Smith2002}{2002, 441}, emphasis added) makes this point even more explicitly: ``By the end of the twentieth century,'' he writes, ``gentrification \emph{fueled by a concerted and systematic partnership of public planning with public and private capital} had moved into the vacuum left by the end of liberal urban policy.''

This analysis of the state-led nature of gentrification is echoed in the popular press, where journalists often draw links between state action and increased rents. When discussing the rezoning of the Inwood neighborhood in Manhattan, for instance, the \emph{New York Times} wrote that ``After the rezoning plan was announced in 2013, years before it was enacted, real estate investors swooped into Inwood and bought more than \$610 million in properties'' (\protect\hyperlink{ref-Haag2019}{Haag 2019}). Public infrastructure programs, such as the BeltLine Park in Atlanta, are also led by the state and feared by residents for their potentially displacing effect (\protect\hyperlink{ref-Lartey2018}{Lartey 2018}). Whether the state is explicitly subsidizing development in an area by providing improved public goods or implicitly subsidizing it by allowing for higher-intensity development, the state's involvement is clear. It is perhaps unsurprising, then, that many communities where state investment has historically been virtually nonexistent are fighting back against plans that would increase financial investment in their neighborhoods (e.g. \protect\hyperlink{ref-Lees2018}{Lees, Annunziata, and Rivas-Alonso 2018}).

The state, then, is inherently involved in the gentrification of American cities. State (in)action led to under-investment in these neighborhoods in the 20th century, and state action is now paving the way for private interests to capitalize on that history. And yet, the existing work on gentrification has largely ignored how---or whether---residents hold the state accountable for their changed life circumstances. Scholars have instead focused on the non-state consequences of gentrification, asking questions primarily about whether gentrification leads to displacement (\protect\hyperlink{ref-Hwang2020}{Hwang and Ding 2020}), how it effects education (\protect\hyperlink{ref-Keels2013}{Keels, Burdick--Will, and Keene 2013}), whether it increases employment (\protect\hyperlink{ref-Meltzer2017}{Meltzer and Ghorbani 2017}), its effects on crime (\protect\hyperlink{ref-Papachristos2011}{Papachristos et al. 2011}), and other matters. While these questions certainly have merit, understanding whether residents turn to the ballot to contest gentrification is of key importance given the role of the state in its mechanics.

\hypertarget{the-potentially-politicizing-nature-of-gentrification}{%
\section*{The Potentially Politicizing Nature of Gentrification}\label{the-potentially-politicizing-nature-of-gentrification}}
\addcontentsline{toc}{section}{The Potentially Politicizing Nature of Gentrification}

\hypertarget{policy-threat}{%
\subsection*{Policy Threat}\label{policy-threat}}
\addcontentsline{toc}{subsection}{Policy Threat}

A growing body of literature in political science and sociology documents how citizens respond when they face threatening circumstances. While sociologists have historically focused on ``extra-institutional'' political behavior such as protest and social movement participation, political scientists have often focused more narrowly on political \emph{outcomes} (\protect\hyperlink{ref-Barrie2021}{Barrie 2021}). Both disciplines, however, provide insights into how political behavior is shaped by government action.

Throughout this literature runs the notion of governmental or policy threat: namely, when individuals feel that they or a group to which they belong are being targeted by government policy, they can be mobilized to take action (\protect\hyperlink{ref-TamCho2006a}{Tam Cho, Gimpel, and Wu 2006}). Van Stekelenburg and Klandermans (\protect\hyperlink{ref-vanStekelenburg2013}{2013, 897}) explain: ``The more people feel that interests of the group and/or principles that the group values are threatened, the angrier they are and the more they are prepared to take part in protest to protect their interests and principles and/or to express their anger.'' \protect\hyperlink{ref-Almeida2018}{Almeida} (\protect\hyperlink{ref-Almeida2018}{2018}) characterizes these threats from the state as ``rights eroding'' threats, noting that threats to abortion and welfare programs often spur citizens to take action. \protect\hyperlink{ref-Reese2011}{Reese} (\protect\hyperlink{ref-Reese2011}{2011}) similarly documents how citizens faced with President Bill Clinton's Personal Responsibility and Work Opportunity Reconciliation Act of 1996, which threatened to radically undermine the U.S.'s social safety net, were mobilized into resistance. Key to this mobilization is the understanding that group-level processes are at work: \protect\hyperlink{ref-Piven1979}{Piven and Cloward} (\protect\hyperlink{ref-Piven1979}{1979}) and \protect\hyperlink{ref-Schlozman1979}{Schlozman and Verba} (\protect\hyperlink{ref-Schlozman1979}{1979}), for instance, document how workers become politically engaged when they understand their unemployment as a widespread phenomenon that demands collective government response and not an individual failure.

\protect\hyperlink{ref-Zepeda-Millan2016}{Zepeda-Millán} (\protect\hyperlink{ref-Zepeda-Millan2016}{2016}) offers key insight into how even ``unconventional'' political actors can be mobilized to participate under certain circumstances. He uses the case of Latinos in Fort Myers, Florida, to demonstrate these processes. In 2006, Americans across the country took to the streets to protest \emph{The Border Protection, Antiterrorism, and Illegal Immigration Control Act of 2005} (H.R. 4437), widely considered to be discriminatory toward Latinos. The bill sought to change the penalty for being an undocumented immigrant from a misdemeanor to a felony, and to provide vast new resources for border enforcement. Zepeda-Millán (\protect\hyperlink{ref-Zepeda-Millan2016}{2016, 270}) shows how neighborhood context and the cultivation of shared Latino identities led ``unconventional political actors---from local soccer league players and nannies to farmworkers and ethnic entrepreneurs---to utilize pre-existing neighborhood assets for the purpose of immigrant mass mobilization.''

These policy threats have also been shown to induce citizens to cast ballots at higher rates. Ariel \protect\hyperlink{ref-White2016}{White} (\protect\hyperlink{ref-White2016}{2016}), for instance, documents how strict immigration enforcement led to higher turnout rates among Latinos in counties across the country. Similarly, \protect\hyperlink{ref-Towler2018}{Towler and Parker} (\protect\hyperlink{ref-Towler2018}{2018}) argues that Black Americans turned out at high rates in the 2016 president because of the threat posed to their community by Donald Trump. Tam Cho and colleagues (\protect\hyperlink{ref-TamCho2006a}{2006, 978}), in a study demonstrating higher turnout among Arab Americans who feel threatened, sum up: ``A solid body of evidence\ldots{} indicates that political mobilization is a direct response to the degree of threat and discrimination a group experiences.''

Gentrification can be understood through this same lens; as discussed above, the state plays a central role in the facilitation of gentrification. Although \protect\hyperlink{ref-Thorpe2021}{Thorpe} (\protect\hyperlink{ref-Thorpe2021}{2021}) argues that scholars have not historically acknowledged the importance of law and policy in the easing of gentrification, the scholars and activists have long written about gentrification using the language of rights. Since Henri Lefebvre's 1968 \emph{Le Droit à la Ville} (\protect\hyperlink{ref-Lefebvre1968}{1968}), scholars and activists have counterposed the demands of capital with citizens' ``right to the city'' and, more recently, ``right to stay put'' (see, for instance, \protect\hyperlink{ref-Hartman2002}{Hartman 2002}). Insofar as residents understand gentrification as an infringements on their \emph{right} to their home or community, it likely activates many of the same responses that have been identified in other contexts of policy threat. Given the highly racialized nature of contemporary gentrification (\protect\hyperlink{ref-Freeman2015}{Freeman and Cai 2015}), nonwhite residents may further feel targeted due to the racial identities of their communities.

\protect\hyperlink{ref-Betancur2002}{Betancur} (\protect\hyperlink{ref-Betancur2002}{2002}) explores how these processes play out in the context of West Town (a collection of neighborhoods in Chicago) in the early 1990s and 2000. Betancur notes that, ``Concerned about improving the tax base, city hall did all it could to promote gentrification.'' Ultimately, ``A crucial element at work in West Town was public-sector support for the processes and institutions that made use of public powers of social control to make life miserable for minority low-income residents'' (806). Betancur argues that the state's role in gentrification was contested politically through both institutional and non-institutional means. Gentrification was a major issue in alderman elections in 1999, following the splitting of the area into 4 wards in the aftermath of the 1989 census. But Betancur also shows how the political process played out in non-institutional spaces as well. To take one example, ``Puerto Ricans made great advances in controlling local schools, getting local institutions (e.g., hospitals and churches) to pay attention to their needs
and cultural and economic realities, promoting community identity and pride, and gaining respect from political forces'' (802).

Similarly, \protect\hyperlink{ref-Martin2007a}{Martin} (\protect\hyperlink{ref-Martin2007a}{2007}) shows how communities in Atlanta, Georgia, resist political displacement in the face of gentrification. Worried about the rising numbers of newcomers to their neighborhoods, long-term residents created or invested in organizations to represent their interests. Martin describes how new and existing residents of the Belleview neighborhood struggled for the upper political hand: ``For seven years, long-time residents and new residents engaged each other in a series of bitter conflicts over political influence. Through separate organizations, they challenged each other about new neighborhood amenities such as a neighborhood library, a grocery store, and streetscaping in the neighborhood business district. They also sparred over valid means
to control crime in the neighborhood, over appropriate neighborhood leadership, and even over the process of gentrification itself'' (616). It is clear that, at least under certain circumstances, long-term residents in gentrifying neighborhoods are able to organize politically---in an electoral sense or otherwise---to contest the gentrification process.

\hypertarget{social-cohesion}{%
\subsection*{Social Cohesion}\label{social-cohesion}}
\addcontentsline{toc}{subsection}{Social Cohesion}

While the policy threat literature indicates that gentrification might lead to \emph{higher} turnout as a community feels itself threatened by the state and by real estate capital, other work indicates that gentrification could \emph{reduce} turnout by undermining social solidarity within a neighborhood. In fact, the lone paper of which I am aware exploring the turnout effects of gentrification (\protect\hyperlink{ref-Knotts2006}{Knotts and Haspel 2006}) comes to this conclusion, finding that turnout for long-term residents in gentrified neighborhoods in Atlanta was lower than long-term residents residing elsewhere. However, because they use only a single snapshot of the registered voter file and compare all voters in gentrified neighborhoods to all voters elsewhere, any causal interpretation of these results is suspect.

Despite limitations in their quantitative analysis, \protect\hyperlink{ref-Knotts2006}{Knotts and Haspel} (\protect\hyperlink{ref-Knotts2006}{2006}) provides a helpful theoretical overview of how the neighborhood and social disruption that attends gentrification can reduce voter participation. This is in line with work such as \protect\hyperlink{ref-Levine2018}{Levine et al.} (\protect\hyperlink{ref-Levine2018}{2018}), which investigated the influence of neighborhood stability---measured using home ownership rates, residential move rates, and median age---was positively associated with political participation in Boston.

More generally, individuals with stronger ties to their communities generally vote at higher rates. \protect\hyperlink{ref-Verba1995}{Verba, Schlozman, and Brady} (\protect\hyperlink{ref-Verba1995}{1995})`s civic voluntarism model indicates that voters' voluntary associations, churches, and workplaces provide them with the tools to participate in democracy. To be sure, political scientists have long argued that there is a social pressure to vote (e.g. \protect\hyperlink{ref-Riker1968}{Riker and Ordeshook 1968}), and there is evidence that citizens participate at higher rates when they fear the opprobrium of their neighbors. \protect\hyperlink{ref-Gerber2008}{Gerber, Green, and Larimer} (\protect\hyperlink{ref-Gerber2008}{2008}) provides one famous example, in which potential voters were assigned to one of four treatment groups. Being reminded that voting is a civic duty, informing voters that their behavior would be studied by researchers, or being told the turnout of members of the household all increased turnout modestly. But the final treatment---being informed of the turnout of ones neighbors, and an implication that the voter's neighbors would be informed about whether they voted---increased turnout by more than 8 percentage points. In short, the social pressure to vote is considered one of the strongest determinants of whether a citizen casts a ballot.

These social ties, however, are apparently weaker in gentrifying / gentrified neighborhoods. Gentrification can reduce the strength of local churches (\protect\hyperlink{ref-Holmes2020}{Holmes 2020}); the same is true of local businesses (\protect\hyperlink{ref-Zukin2009}{Zukin et al. 2009}). These are the sorts of places where social capital is built; it is perhaps unsurprising, then, Newman, Velez, and Pearson-Merkowitz (\protect\hyperlink{ref-Newman2016}{2016, 340}) concludes that ``gentrification `loosens' the social fabric of black communities'' and results in lower trust in neighbors and lower reported political engagement, although they recognize the need for ``longitudinal and experimental research designs,'' which their work lacks (341). Similarly, \protect\hyperlink{ref-Betancur2011}{Betancur} (\protect\hyperlink{ref-Betancur2011}{2011}) traces the dissolution of social network and local institutions in Latino communities in Chicago undergoing gentrification. Gentrification clearly poses a serious threat to the social capital long associated with higher civic participation and may thus lead to lower turnout.

\hypertarget{studies-specifically-on-this}{%
\subsection*{Studies specifically on this}\label{studies-specifically-on-this}}
\addcontentsline{toc}{subsection}{Studies specifically on this}

\hypertarget{administrative-data-to-explore-gentrification}{%
\section*{Administrative Data to Explore Gentrification}\label{administrative-data-to-explore-gentrification}}
\addcontentsline{toc}{section}{Administrative Data to Explore Gentrification}

Although this project is primarily about the potentially politicizing effect of remaining in, or leaving, a gentrifying neighborhood, it also joins recent scholarship that uses administrative data to understand how gentrification shapes residential mobility patterns in a given region. While much of the early work on gentrification relied heavily on survey work

\hypertarget{data-and-methods}{%
\section*{Data and Methods}\label{data-and-methods}}
\addcontentsline{toc}{section}{Data and Methods}

\hypertarget{why-atlanta-pros-and-cons-of-voter-file-data}{%
\subsection*{Why Atlanta / Pros and Cons of Voter File Data}\label{why-atlanta-pros-and-cons-of-voter-file-data}}
\addcontentsline{toc}{subsection}{Why Atlanta / Pros and Cons of Voter File Data}

\hypertarget{definition-of-gentrification}{%
\subsection*{Definition of Gentrification}\label{definition-of-gentrification}}
\addcontentsline{toc}{subsection}{Definition of Gentrification}

\hypertarget{probably-a-matched-twfe}{%
\subsection*{Probably a matched TWFE}\label{probably-a-matched-twfe}}
\addcontentsline{toc}{subsection}{Probably a matched TWFE}

\newpage

\hypertarget{references}{%
\section*{References}\label{references}}
\addcontentsline{toc}{section}{References}

\hypertarget{refs}{}
\begin{CSLReferences}{1}{0}
\leavevmode\hypertarget{ref-Almeida2018}{}%
Almeida, Paul D. 2018. {``The {Role} of {Threat} in {Collective Action}.''} In \emph{The {Wiley Blackwell Companion} to {Social Movements}}, edited by David A. Snow, Sarah A. Soule, Hanspeter Kriesi, and Holly J. McCammon, 43--62. {John Wiley \& Sons, Ltd}. \url{https://doi.org/10.1002/9781119168577.ch2}.

\leavevmode\hypertarget{ref-Barrie2021}{}%
Barrie, Christopher. 2021. {``Political Sociology in a Time of Protest.''} \emph{Current Sociology}, June, 00113921211024692. \url{https://doi.org/10.1177/00113921211024692}.

\leavevmode\hypertarget{ref-Betancur2002}{}%
Betancur, John. 2002. {``The {Politics} of {Gentrification}: {The Case} of {West Town} in {Chicago}.''} \emph{Urban Affairs Review} 37 (6): 780--814. \url{https://doi.org/10.1177/107874037006002}.

\leavevmode\hypertarget{ref-Betancur2011}{}%
---------. 2011. {``Gentrification and {Community Fabric} in {Chicago}.''} \emph{Urban Studies} 48 (2): 383--406. \url{https://doi.org/10.1177/0042098009360680}.

\leavevmode\hypertarget{ref-Birch2009}{}%
Birch, Eugénie L. 2009. {``Downtown in the {`{New American City}'}.''} \emph{The ANNALS of the American Academy of Political and Social Science} 626 (1): 134--53. \url{https://doi.org/10.1177/0002716209344169}.

\leavevmode\hypertarget{ref-Freeman2015}{}%
Freeman, Lance, and Tiancheng Cai. 2015. {``White {Entry} into {Black Neighborhoods}: {Advent} of a {New Era}?''} \emph{The Annals of the American Academy of Political and Social Science} 660: 302--18. \url{http://www.jstor.org/stable/24541839}.

\leavevmode\hypertarget{ref-Gerber2008}{}%
Gerber, Alan S., Donald P. Green, and Christopher W. Larimer. 2008. {``Social {Pressure} and {Voter Turnout}: {Evidence} from a {Large}-{Scale Field Experiment}.''} \emph{American Political Science Review} 102 (1): 33--48. \url{https://doi.org/10.1017/S000305540808009X}.

\leavevmode\hypertarget{ref-Haag2019}{}%
Haag, Matthew. 2019. {``It's {Manhattan}'s {Last Affordable Neighborhood}. {But} for {How Long}?''} \emph{The New York Times: New York}, September 27, 2019. \url{https://www.nytimes.com/2019/09/27/nyregion/its-manhattans-last-affordable-neighborhood-but-for-how-long.html}.

\leavevmode\hypertarget{ref-Hartman2002}{}%
Hartman, Chester W. 2002. \emph{Between Eminence and Notoriety: Four Decades of Radical Urban Planning}. {New Brunswick, N.J}: {Center for Urban Policy Research}.

\leavevmode\hypertarget{ref-Holmes2020}{}%
Holmes, Kristin E. 2020. {``Religious Agency in the Dynamics of Gentrification: {Moving} in, Moving Out, and Staying Put in {Philadelphia}.''} In \emph{The {Routledge Handbook} of {Religion} and {Cities}}. {Routledge}.

\leavevmode\hypertarget{ref-Hwang2020}{}%
Hwang, Jackelyn, and Lei Ding. 2020. {``Unequal {Displacement}: {Gentrification}, {Racial Stratification}, and {Residential Destinations} in {Philadelphia}.''} \emph{American Journal of Sociology} 126 (2): 354--406. \url{https://doi.org/10.1086/711015}.

\leavevmode\hypertarget{ref-Hyra2015}{}%
Hyra, Derek. 2015. {``The Back-to-the-City Movement: {Neighbourhood} Redevelopment and Processes of Political and Cultural Displacement.''} \emph{Urban Studies} 52 (10): 1753--73. \url{https://doi.org/10.1177/0042098014539403}.

\leavevmode\hypertarget{ref-Jackson1985}{}%
Jackson, Kenneth T. 1985. \emph{Crabgrass Frontier: The Suburbanization of the {United States}}. {New York}: {Oxford University Press}.

\leavevmode\hypertarget{ref-Keels2013}{}%
Keels, Micere, Julia Burdick--Will, and Sara Keene. 2013. {``The {Effects} of {Gentrification} on {Neighborhood Public Schools}.''} \emph{City \& Community} 12 (3): 238--59. \url{https://doi.org/10.1111/cico.12027}.

\leavevmode\hypertarget{ref-Knotts2006}{}%
Knotts, H. Gibbs, and Moshe Haspel. 2006. {``The {Impact} of {Gentrification} on {Voter Turnout}.''} \emph{Social Science Quarterly} 87 (1): 110--21. \url{http://www.jstor.org/stable/42956112}.

\leavevmode\hypertarget{ref-Lartey2018}{}%
Lartey, Jamiles. 2018. {``Nowhere for People to Go: Who Will Survive the Gentrification of {Atlanta}?''} \emph{The Guardian: Cities}, October 23, 2018. \url{http://www.theguardian.com/cities/2018/oct/23/nowhere-for-people-to-go-who-will-survive-the-gentrification-of-atlanta}.

\leavevmode\hypertarget{ref-Lees2018}{}%
Lees, Loretta, Sandra Annunziata, and Clara Rivas-Alonso. 2018. {``Resisting {Planetary Gentrification}: {The Value} of {Survivability} in the {Fight} to {Stay Put}.''} \emph{Annals of the American Association of Geographers} 108 (2): 346--55. \url{https://doi.org/10.1080/24694452.2017.1365587}.

\leavevmode\hypertarget{ref-Lefebvre1968}{}%
Lefebvre, Henri. 1968. \emph{Le {Droit} à La {Ville}}. {Paris}: {Anthropos}.

\leavevmode\hypertarget{ref-Levine2018}{}%
Levine, Jeremy R., Theodore S. Leenman, Carl Gershenson, and David M. Hureau. 2018. {``Political {Places}: {Neighborhood Social Organization} and the {Ecology} of {Political Behaviors}.''} \emph{Social Science Quarterly} 99 (1): 201--15. \url{https://doi.org/10.1111/ssqu.12352}.

\leavevmode\hypertarget{ref-Martin2007a}{}%
Martin, Leslie. 2007. {``Fighting for {Control}: {Political Displacement} in {Atlanta}'s {Gentrifying Neighborhoods}.''} \emph{Urban Affairs Review} 42 (5): 603--28. \url{https://doi.org/10.1177/1078087406296604}.

\leavevmode\hypertarget{ref-Massey2003}{}%
Massey, Douglas S., and Nancy A. Denton. 2003. \emph{American Apartheid: Segregation and the Making of the Underclass}. 10. print. {Cambridge, Mass.}: {Harvard Univ. Press}.

\leavevmode\hypertarget{ref-Meltzer2017}{}%
Meltzer, Rachel, and Pooya Ghorbani. 2017. {``Does Gentrification Increase Employment Opportunities in Low-Income Neighborhoods?''} \emph{Regional Science and Urban Economics} 66 (September): 52--73. \url{https://doi.org/10.1016/j.regsciurbeco.2017.06.002}.

\leavevmode\hypertarget{ref-Newman2016}{}%
Newman, Benjamin J., Yamil Velez, and Shanna Pearson-Merkowitz. 2016. {``Diversity of a {Different Kind}: {Gentrification} and {Its Impact} on {Social Capital} and {Political Participation} in {Black Communities}.''} \emph{Journal of Race, Ethnicity, and Politics} 1 (2): 316--47. \url{https://doi.org/10.1017/rep.2016.8}.

\leavevmode\hypertarget{ref-Papachristos2011}{}%
Papachristos, Andrew V., Chris M. Smith, Mary L. Scherer, and Melissa A. Fugiero. 2011. {``More {Coffee}, {Less Crime}? {The Relationship} Between {Gentrification} and {Neighborhood Crime Rates} in {Chicago}, 1991 to 2005.''} \emph{City \& Community} 10 (3): 215--40. \url{https://doi.org/10.1111/j.1540-6040.2011.01371.x}.

\leavevmode\hypertarget{ref-Phillips-Fein2017}{}%
Phillips-Fein, Kim. 2017. \emph{Fear City: {New York}'s Fiscal Crisis and the Rise of Austerity Politics}. First edition. {New York}: {Metropolitan Books}.

\leavevmode\hypertarget{ref-Piven1979}{}%
Piven, Frances Fox, and Richard A. Cloward. 1979. \emph{Poor People's Movements: Why They Succeed, How They Fail}. {New York}: {Vintage books}.

\leavevmode\hypertarget{ref-Reese2011}{}%
Reese, Ellen. 2011. \emph{They {Say Cutback}, {We Say Fight Back}!: {Welfare Activism} in an {Era} of {Retrenchment}}. {New York, UNITED STATES}: {Russell Sage Foundation}. \url{http://ebookcentral.proquest.com/lib/nyulibrary-ebooks/detail.action?docID=4386937}.

\leavevmode\hypertarget{ref-Riker1968}{}%
Riker, William H., and Peter C. Ordeshook. 1968. {``A {Theory} of the {Calculus} of {Voting}.''} \emph{The American Political Science Review} 62 (1): 25--42. \url{https://doi.org/10.2307/1953324}.

\leavevmode\hypertarget{ref-Rothstein2017}{}%
Rothstein, Richard. 2017. \emph{The Color of Law: A Forgotten History of How Our Government Segregated {America}}. First edition. {New York ; London}: {Liveright Publishing Corporation, a division of W. W. Norton \& Company}.

\leavevmode\hypertarget{ref-Schlozman1979}{}%
Schlozman, Kay Kehman, and Sidney Verba. 1979. \emph{Injury to {Insult}: {Unemployment}, {Class}, and {Political Response}}. {Harvard University Press}. \url{http://books.google.com?id=m6XTdRlx3Q8C}.

\leavevmode\hypertarget{ref-Smith2002}{}%
Smith, Neil. 2002. {``New {Globalism}, {New Urbanism}: {Gentrification} as {Global Urban Strategy}.''} \emph{Antipode} 34 (3): 427--50. \url{https://doi.org/10.1111/1467-8330.00249}.

\leavevmode\hypertarget{ref-Stein2019}{}%
Stein, Samuel. 2019. \emph{Capital City: Gentrification and the Real Estate State}. Jacobin Series. {London ; Brooklyn, NY}: {Verso}.

\leavevmode\hypertarget{ref-vanStekelenburg2013}{}%
Stekelenburg, Jacquelien van, and Bert Klandermans. 2013. {``The Social Psychology of Protest.''} \emph{Current Sociology} 61 (5-6): 886--905. \url{https://doi.org/10.1177/0011392113479314}.

\leavevmode\hypertarget{ref-Sugrue1998}{}%
Sugrue, Thomas J. 1998. \emph{The Origins of the Urban Crisis: Race and Inequality in Postwar {Detroit}}. 1st paperback ed. Princeton Studies in {American} Politics. {Princeton, N.J}: {Princeton University Press}.

\leavevmode\hypertarget{ref-TamCho2006a}{}%
Tam Cho, Wendy K., James G. Gimpel, and Tony Wu. 2006. {``Clarifying the {Role} of {SES} in {Political Participation}: {Policy Threat} and {Arab American Mobilization}.''} \emph{Journal of Politics} 68 (4): 977--91. \url{https://doi.org/10.1111/j.1468-2508.2006.00482.x}.

\leavevmode\hypertarget{ref-Taylor2019}{}%
Taylor, Keeanga-Yamahtta. 2019. \emph{Race for Profit: How Banks and the Real Estate Industry Undermined Black Homeownership}. Justice, Power, and Politics. {Chapel Hill}: {The University of North Carolina Press}.

\leavevmode\hypertarget{ref-Thorpe2021}{}%
Thorpe, Amelia. 2021. {``Regulatory Gentrification: {Documents}, Displacement and the Loss of Low-Income Housing.''} \emph{Urban Studies} 58 (13): 2623--39. \url{https://doi.org/10.1177/0042098020960569}.

\leavevmode\hypertarget{ref-Towler2018}{}%
Towler, Christopher C., and Christopher S. Parker. 2018. {``Between {Anger} and {Engagement}: {Donald Trump} and {Black America}.''} \emph{Journal of Race, Ethnicity, and Politics} 3 (1): 219--53. \url{https://doi.org/10.1017/rep.2017.38}.

\leavevmode\hypertarget{ref-Verba1995}{}%
Verba, Sidney, Kay Lehman Schlozman, and Henry E. Brady. 1995. \emph{Voice and {Equality}: {Civic Voluntarism} in {American Politics}}. {Harvard University Press}. \url{http://books.google.com?id=RUkvEAAAQBAJ}.

\leavevmode\hypertarget{ref-White2016}{}%
White, Ariel. 2016. {``When {Threat Mobilizes}: {Immigration Enforcement} and {Latino Voter Turnout}.''} \emph{Political Behavior} 38 (2): 355--82. \url{https://doi.org/10.1007/s11109-015-9317-5}.

\leavevmode\hypertarget{ref-Zepeda-Millan2016}{}%
Zepeda-Millán, Chris. 2016. {``Weapons of the ({Not So}) {Weak}: {Immigrant Mass Mobilization} in the {US South}.''} \emph{Critical Sociology} 42 (2): 269--87. \url{https://doi.org/10.1177/0896920514527846}.

\leavevmode\hypertarget{ref-Zukin2009}{}%
Zukin, Sharon, Valerie Trujillo, Peter Frase, Danielle Jackson, Tim Recuber, and Abraham Walker. 2009. {``New {Retail Capital} and {Neighborhood Change}: {Boutiques} and {Gentrification} in {New York City}.''} \emph{City \& Community} 8 (1): 47--64. \url{https://doi.org/10.1111/j.1540-6040.2009.01269.x}.

\end{CSLReferences}

\end{document}
