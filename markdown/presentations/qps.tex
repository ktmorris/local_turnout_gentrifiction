% Options for packages loaded elsewhere
\PassOptionsToPackage{unicode}{hyperref}
\PassOptionsToPackage{hyphens}{url}
%
\documentclass[
  ignorenonframetext,
  aspectratio=169]{beamer}
\usepackage{pgfpages}
\setbeamertemplate{caption}[numbered]
\setbeamertemplate{caption label separator}{: }
\setbeamercolor{caption name}{fg=normal text.fg}
\beamertemplatenavigationsymbolsempty
% Prevent slide breaks in the middle of a paragraph
\widowpenalties 1 10000
\raggedbottom
\setbeamertemplate{part page}{
  \centering
  \begin{beamercolorbox}[sep=16pt,center]{part title}
    \usebeamerfont{part title}\insertpart\par
  \end{beamercolorbox}
}
\setbeamertemplate{section page}{
  \centering
  \begin{beamercolorbox}[sep=12pt,center]{part title}
    \usebeamerfont{section title}\insertsection\par
  \end{beamercolorbox}
}
\setbeamertemplate{subsection page}{
  \centering
  \begin{beamercolorbox}[sep=8pt,center]{part title}
    \usebeamerfont{subsection title}\insertsubsection\par
  \end{beamercolorbox}
}
\AtBeginPart{
  \frame{\partpage}
}
\AtBeginSection{
  \ifbibliography
  \else
    \frame{\sectionpage}
  \fi
}
\AtBeginSubsection{
  \frame{\subsectionpage}
}
\usepackage{amsmath,amssymb}
\usepackage{lmodern}
\usepackage{ifxetex,ifluatex}
\ifnum 0\ifxetex 1\fi\ifluatex 1\fi=0 % if pdftex
  \usepackage[T1]{fontenc}
  \usepackage[utf8]{inputenc}
  \usepackage{textcomp} % provide euro and other symbols
\else % if luatex or xetex
  \usepackage{unicode-math}
  \defaultfontfeatures{Scale=MatchLowercase}
  \defaultfontfeatures[\rmfamily]{Ligatures=TeX,Scale=1}
\fi
\usetheme[]{Berlin}
% Use upquote if available, for straight quotes in verbatim environments
\IfFileExists{upquote.sty}{\usepackage{upquote}}{}
\IfFileExists{microtype.sty}{% use microtype if available
  \usepackage[]{microtype}
  \UseMicrotypeSet[protrusion]{basicmath} % disable protrusion for tt fonts
}{}
\makeatletter
\@ifundefined{KOMAClassName}{% if non-KOMA class
  \IfFileExists{parskip.sty}{%
    \usepackage{parskip}
  }{% else
    \setlength{\parindent}{0pt}
    \setlength{\parskip}{6pt plus 2pt minus 1pt}}
}{% if KOMA class
  \KOMAoptions{parskip=half}}
\makeatother
\usepackage{xcolor}
\IfFileExists{xurl.sty}{\usepackage{xurl}}{} % add URL line breaks if available
\IfFileExists{bookmark.sty}{\usepackage{bookmark}}{\usepackage{hyperref}}
\hypersetup{
  pdftitle={There Votes the Neighborhood},
  pdfauthor={Kevin Morris},
  hidelinks,
  pdfcreator={LaTeX via pandoc}}
\urlstyle{same} % disable monospaced font for URLs
\newif\ifbibliography
\setlength{\emergencystretch}{3em} % prevent overfull lines
\providecommand{\tightlist}{%
  \setlength{\itemsep}{0pt}\setlength{\parskip}{0pt}}
\setcounter{secnumdepth}{-\maxdimen} % remove section numbering
\ifluatex
  \usepackage{selnolig}  % disable illegal ligatures
\fi
\newlength{\cslhangindent}
\setlength{\cslhangindent}{1.5em}
\newlength{\csllabelwidth}
\setlength{\csllabelwidth}{3em}
\newenvironment{CSLReferences}[2] % #1 hanging-ident, #2 entry spacing
 {% don't indent paragraphs
  \setlength{\parindent}{0pt}
  % turn on hanging indent if param 1 is 1
  \ifodd #1 \everypar{\setlength{\hangindent}{\cslhangindent}}\ignorespaces\fi
  % set entry spacing
  \ifnum #2 > 0
  \setlength{\parskip}{#2\baselineskip}
  \fi
 }%
 {}
\usepackage{calc}
\newcommand{\CSLBlock}[1]{#1\hfill\break}
\newcommand{\CSLLeftMargin}[1]{\parbox[t]{\csllabelwidth}{#1}}
\newcommand{\CSLRightInline}[1]{\parbox[t]{\linewidth - \csllabelwidth}{#1}\break}
\newcommand{\CSLIndent}[1]{\hspace{\cslhangindent}#1}

\title{There Votes the Neighborhood}
\subtitle{Gentrification, Residential Mobility, and Political
Participation in Atlanta}
\author{Kevin Morris}
\date{}
\institute{CUNY Graduate Center, Sociology}

\begin{document}
\frame{\titlepage}
\begin{abstract}
\href{mailto:kevin.morris@nyu.edu}{\nolinkurl{kevin.morris@nyu.edu}}
\end{abstract}

\begin{frame}{Gentrification as a State Practice}
\protect\hypertarget{gentrification-as-a-state-practice}{}
\begin{itemize}[<+->]
\tightlist
\item
  In recent years, scholars have increasingly noted the role played by
  the state in both the economic deterioration of American cities in the
  second half of the 20th century (e.g. Sugrue 1998) and the inflows of
  capital and agents of displacement in the 21st (e.g. Stein 2019).
\end{itemize}

\begin{itemize}[<+->]
\tightlist
\item
  Fights over rezonings in places like New York City (Haag 2019), and
  improved public goods in places like Atlanta (Lartey 2018) have thrown
  the consequences of the state's (in)action into stark relief.
\end{itemize}

\begin{itemize}[<+->]
\tightlist
\item
  Although gentrification is not possible without participation from the
  state, scholars have not investigated how the experience of
  gentrification influences citizen identity formation and political
  participation (Thorpe 2021; but see Knotts and Haspel 2006).
\end{itemize}
\end{frame}

\begin{frame}{Political Threat}
\protect\hypertarget{political-threat}{}
\begin{itemize}[<+->]
\tightlist
\item
  On the one hand, sociological and political science theory predicts
  that gentrification would \emph{increase} political participation.
\end{itemize}

\begin{itemize}[<+->]
\tightlist
\item
  Political threat can lead citizens to participate at higher rates
  (e.g. Zepeda-Millán 2016).
\end{itemize}

\begin{itemize}[<+->]
\tightlist
\item
  A growing body of qualitative work explores how local communities form
  networks capable of resisting displacement through engagement with
  local government (Betancur 2002; Martin 2007).
\end{itemize}
\end{frame}

\begin{frame}{Social Cohesion}
\protect\hypertarget{social-cohesion}{}
\begin{itemize}[<+->]
\tightlist
\item
  On the other hand, political scientists have long noted that
  individuals who feel strongly connected to their communities
  participate at higher rates (Riker and Ordeshook 1968; Verba,
  Schlozman, and Brady 1995).
\end{itemize}

\begin{itemize}[<+->]
\tightlist
\item
  Gentrification can lead to lower levels of social cohesion (Zukin et
  al. 2009; Holmes 2020).
\end{itemize}
\end{frame}

\begin{frame}{Empirical Framework}
\protect\hypertarget{empirical-framework}{}
\begin{itemize}[<+->]
\tightlist
\item
  I use multiple geocoded snapshots of the registered voter file in
  Atlanta between 2010 and 2020 to track individuals' mobility and
  participation patterns. Atlanta has a very high registration rate
  (Niesse 2021) and relatively small noncitizen population.
\end{itemize}

\begin{itemize}[<+->]
\tightlist
\item
  I start by identifying \emph{gentrifiable} neighborhoods in 2010. I
  then construct matched pairs of voters, where one lived in a
  neighborhood that went on to gentrify (i.e., was ``treated'' by
  gentrification) while the other's neighborhood did not.
\end{itemize}

\begin{itemize}[<+->]
\tightlist
\item
  I'll investigate mobility patterns \emph{and} effects of experiencing
  gentrification on political participation. These effects will be
  explored separately for individuals who moved out of their 2010
  neighborhood and for those who stayed.
\end{itemize}
\end{frame}

\begin{frame}{Thanks!}
\protect\hypertarget{thanks}{}
\href{mailto:kevin.morris@nyu.edu}{\nolinkurl{kevin.morris@nyu.edu}}
\end{frame}

\begin{frame}[allowframebreaks]{References}
\protect\hypertarget{references}{}
\hypertarget{refs}{}
\begin{CSLReferences}{1}{0}
\leavevmode\hypertarget{ref-Betancur2002}{}%
Betancur, John. 2002. {``The {Politics} of {Gentrification}: The {Case}
of {West Town} in {Chicago}.''} \emph{Urban Affairs Review} 37 (6):
780--814. \url{https://doi.org/10.1177/107874037006002}.

\leavevmode\hypertarget{ref-Haag2019}{}%
Haag, Matthew. 2019. {``It's {Manhattan}'s {Last Affordable
Neighborhood}. {But} for {How Long}?''} \emph{The New York Times: New
York}, September 27, 2019.
\url{https://www.nytimes.com/2019/09/27/nyregion/its-manhattans-last-affordable-neighborhood-but-for-how-long.html}.

\leavevmode\hypertarget{ref-Holmes2020}{}%
Holmes, Kristin E. 2020. {``Religious Agency in the Dynamics of
Gentrification: Moving in, Moving Out, and Staying Put in
{Philadelphia}.''} In \emph{The {Routledge Handbook} of {Religion} and
{Cities}}. {Routledge}.

\leavevmode\hypertarget{ref-Knotts2006}{}%
Knotts, H. Gibbs, and Moshe Haspel. 2006. {``The {Impact} of
{Gentrification} on {Voter Turnout}.''} \emph{Social Science Quarterly}
87 (1): 110--21. \url{http://www.jstor.org/stable/42956112}.

\leavevmode\hypertarget{ref-Lartey2018}{}%
Lartey, Jamiles. 2018. {``Nowhere for People to Go: Who Will Survive the
Gentrification of {Atlanta}?''} \emph{The Guardian: Cities}, October 23,
2018.
\url{http://www.theguardian.com/cities/2018/oct/23/nowhere-for-people-to-go-who-will-survive-the-gentrification-of-atlanta}.

\leavevmode\hypertarget{ref-Martin2007a}{}%
Martin, Leslie. 2007. {``Fighting for {Control}: Political
{Displacement} in {Atlanta}'s {Gentrifying Neighborhoods}.''}
\emph{Urban Affairs Review} 42 (5): 603--28.
\url{https://doi.org/10.1177/1078087406296604}.

\leavevmode\hypertarget{ref-Niesse2021a}{}%
Niesse, Mark. 2021. {``Almost All Eligible {Georgians} Are Registered to
Vote, Data Show.''} \emph{The Atlanta Journal-Constitution: Politics},
July 19, 2021.

\leavevmode\hypertarget{ref-Riker1968}{}%
Riker, William H., and Peter C. Ordeshook. 1968. {``A {Theory} of the
{Calculus} of {Voting}.''} \emph{The American Political Science Review}
62 (1): 25--42. \url{https://doi.org/10.2307/1953324}.

\leavevmode\hypertarget{ref-Stein2019}{}%
Stein, Samuel. 2019. \emph{Capital City: Gentrification and the Real
Estate State}. Jacobin Series. {London ; Brooklyn, NY}: {Verso}.

\leavevmode\hypertarget{ref-Sugrue1998}{}%
Sugrue, Thomas J. 1998. \emph{The Origins of the Urban Crisis: Race and
Inequality in Postwar {Detroit}}. 1st paperback ed. Princeton Studies in
{American} Politics. {Princeton, N.J}: {Princeton University Press}.

\leavevmode\hypertarget{ref-Thorpe2021}{}%
Thorpe, Amelia. 2021. {``Regulatory Gentrification: Documents,
Displacement and the Loss of Low-Income Housing.''} \emph{Urban Studies}
58 (13): 2623--39. \url{https://doi.org/10.1177/0042098020960569}.

\leavevmode\hypertarget{ref-Verba1995}{}%
Verba, Sidney, Kay Lehman Schlozman, and Henry E. Brady. 1995.
\emph{Voice and {Equality}: Civic {Voluntarism} in {American Politics}}.
{Harvard University Press}.
\url{http://books.google.com?id=RUkvEAAAQBAJ}.

\leavevmode\hypertarget{ref-Zepeda-Millan2016}{}%
Zepeda-Millán, Chris. 2016. {``Weapons of the ({Not So}) {Weak}:
Immigrant {Mass Mobilization} in the {US South}.''} \emph{Critical
Sociology} 42 (2): 269--87.
\url{https://doi.org/10.1177/0896920514527846}.

\leavevmode\hypertarget{ref-Zukin2009}{}%
Zukin, Sharon, Valerie Trujillo, Peter Frase, Danielle Jackson, Tim
Recuber, and Abraham Walker. 2009. {``New {Retail Capital} and
{Neighborhood Change}: Boutiques and {Gentrification} in {New York
City}.''} \emph{City \& Community} 8 (1): 47--64.
\url{https://doi.org/10.1111/j.1540-6040.2009.01269.x}.

\end{CSLReferences}
\end{frame}

\end{document}
